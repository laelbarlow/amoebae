\documentclass[12pt,letterpaper]{article}
\usepackage{amsmath}
\usepackage[pdftex]{graphicx}
% Load hyperref package without putting lines around hyperlinks.
%\usepackage[hidelinks]{hyperref}
\PassOptionsToPackage{hyphens}{url}\usepackage[hidelinks]{hyperref}
\usepackage{color} 
\usepackage{xcolor} 
\usepackage{xspace}
\usepackage{anysize}
\usepackage{setspace}
\usepackage{multicol} % This allows multiple columns
\usepackage[nottoc,numbib]{tocbibind} % This makes refs a section
\usepackage[pagewise]{lineno}
\usepackage{tcolorbox} % For making boxes around text.
\usepackage[xindy,toc]{glossaries} % Must come after hyperref package.
%\usepackage{url}
\usepackage[T1]{fontenc} % Makes > not typeset as inverted question mark.

% Remove the "References" header from the bibliography.
\usepackage{etoolbox}
\patchcmd{\thebibliography}{\section*{\refname}}{}{}{}


% Run external commands to generate a tex file containing the output of the
% 'amoebae -h' command.
\immediate\write18{bash generate_amoebae_help_output_text.sh}


% Import stuff for formatting citations.
\usepackage{natbib}

% Format paragraphs.
%\setlength{\parskip}{\baselineskip}%
%\setlength{\parindent}{0pt}%
\setlength{\parindent}{0em}
\setlength{\parskip}{1em}
%\renewcommand{\baselinestretch}{2.0}

% Define command for wrapping large words.
\newcommand*\wrapletters[1]{\wr@pletters#1\@nil}
\def\wr@pletters#1#2\@nil{#1\allowbreak\if&#2&\else\wr@pletters#2\@nil\fi}

% Use the listings package to automatically wrap text (unlike with just using
% verbatim).
\usepackage{listings}
\lstset{
basicstyle=\small\ttfamily,
columns=flexible,
breaklines=true,
keepspaces=true
} 


%% Format section headers (this is not ideal when there are many short
%%subsections. 
%\usepackage[tiny]{titlesec}
%\titlespacing\subsection{0pt}{12pt plus 4pt minus 2pt}{0pt plus 2pt minus 2pt}
%\titlespacing\subsubsection{0pt}{12pt plus 4pt minus 2pt}{0pt plus 2pt minus 2pt}
%\titlespacing\subsubsection{0pt}{12pt plus 4pt minus 2pt}{0pt plus 2pt minus
%2pt}

\marginsize{2.5 cm}{2.5 cm}{1 cm}{1 cm} % Works out to one inch margins.

%\parindent 1cm
\graphicspath{{figures/}}
%\pagenumbering{arabic}

\pagenumbering{roman}

\makeglossaries

\begin{document}
\begin{titlepage}
	\centering
    {\huge AMOEBAE command line interface documentation\par}
	\vspace{2cm}
    {\Large Lael D. Barlow\par}
	\vfill
	{\large Version of \today\par}
\end{titlepage}

%Optional table of contents.
\newpage
\tableofcontents

\newpage
\pagenumbering{arabic}
% Start line numbers on this page
\begin{linenumbers}

\section{Command reference}

Documentation for each AMOEBAE command and the various options may be accessed
from the command line via the "-h" options. The following command reference
information is the output of running amoebae (and each command) with the "-h"
option.

% Import tex file output from the generate_amoebae_help_output_text.sh.
\input{amoebae_help_output.tex}


%\printglossaries % https://en.wikibooks.org/wiki/LaTeX/Glossary

\newpage
% End line numbering.
\end{linenumbers}

%% The unsrt style orders references by appearance, but puts given names first.
%% The plain style orders references alphabetically, but puts surnames first.
%\bibliographystyle{laelstyle5} 
%\begin{multicols}{2}
%{\footnotesize % The footnotesize command makes the text smaller.
%\bibliography{references/AMOEBAE}}
%\end{multicols}


\end{document}


