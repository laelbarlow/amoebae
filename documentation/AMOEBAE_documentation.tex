\documentclass[12pt,letterpaper]{article}
\usepackage{amsmath}
\usepackage[pdftex]{graphicx}
% Load hyperref package without putting lines around hyperlinks.
%\usepackage[hidelinks]{hyperref}
\PassOptionsToPackage{hyphens}{url}\usepackage[hidelinks]{hyperref}
\usepackage{color} 
\usepackage{xcolor} 
\usepackage{xspace}
\usepackage{anysize}
\usepackage{setspace}
\usepackage{multicol} % This allows multiple columns
\usepackage[nottoc,numbib]{tocbibind} % This makes refs a section
\usepackage[pagewise]{lineno}
\usepackage{tcolorbox} % For making boxes around text.
\usepackage[xindy,toc]{glossaries} % Must come after hyperref package.
%\usepackage{url}
\usepackage[T1]{fontenc} % Makes > not typeset as inverted question mark.

% Remove the "References" header from the bibliography.
\usepackage{etoolbox}
\patchcmd{\thebibliography}{\section*{\refname}}{}{}{}


% Run external commands to generate a tex file containing the output of the
% 'amoebae -h' command.
\immediate\write18{bash generate_amoebae_help_output_text.sh}


% Import stuff for formatting citations.
\usepackage{natbib}

% Format paragraphs.
%\setlength{\parskip}{\baselineskip}%
%\setlength{\parindent}{0pt}%
\setlength{\parindent}{0em}
\setlength{\parskip}{1em}
%\renewcommand{\baselinestretch}{2.0}

% Define command for wrapping large words.
\newcommand*\wrapletters[1]{\wr@pletters#1\@nil}
\def\wr@pletters#1#2\@nil{#1\allowbreak\if&#2&\else\wr@pletters#2\@nil\fi}

% Use the listings package to automatically wrap text (unlike with just using
% verbatim).
\usepackage{listings}
\lstset{
basicstyle=\small\ttfamily,
columns=flexible,
breaklines=true,
keepspaces=true
} 


%% Format section headers (this is not ideal when there are many short
%%subsections. 
%\usepackage[tiny]{titlesec}
%\titlespacing\subsection{0pt}{12pt plus 4pt minus 2pt}{0pt plus 2pt minus 2pt}
%\titlespacing\subsubsection{0pt}{12pt plus 4pt minus 2pt}{0pt plus 2pt minus 2pt}
%\titlespacing\subsubsection{0pt}{12pt plus 4pt minus 2pt}{0pt plus 2pt minus
%2pt}

\marginsize{2.5 cm}{2.5 cm}{1 cm}{1 cm} % Works out to one inch margins.

%\parindent 1cm
\graphicspath{{figures/}}
%\pagenumbering{arabic}

\pagenumbering{roman}

\makeglossaries

\begin{document}
\begin{titlepage}
	\centering
    {\huge AMOEBAE documentation\par}
	\vspace{2cm}
    {\Large Lael D. Barlow\par}
	\vfill
	{\large Version of \today\par}
\end{titlepage}

%Optional table of contents.
\newpage
\tableofcontents

\newpage
\pagenumbering{arabic}
% Start line numbers on this page
\begin{linenumbers}

\section{Introduction}


\subsection{What is AMOEBAE?}

Analysis of MOlecular Evolution with BAtch Entry (AMOEBAE) is a bioinformatics
    software toolkit composed primarily of scripts written in the Python3
    language.  AMOEBAE scripts use existing Python packages including Biopython
    \citep{cock2009}, the Environment for Tree Exploration (ETE3)
    \citep{huerta-cepas2016}, pandas, and Matplotlib \citep{hunter2007} for
    setting up, running, and summarizing analyses of molecular evolution using
    bioinformatics software packages including MUSCLE \citep{edgar2004}, BLAST+
    \citep{camacho2009}, HMMer3 \citep{eddy1998}, and IQ-Tree
    \citep{nguyen2015}. Applications include identifying and classifying
    predicted peptide sequences according to their evolutionary relationships
    with homologues. All dependencies are freely available, and AMOEBAE code is
    open-source (see \autoref*{license_section}) and available on GitHub
    (\url{https://github.com/laelbarlow/amoebae}). 

\subsection{Why AMOEBAE?}

    Webservices such as those provided by NCBI
    (\url{https://blast.ncbi.nlm.nih.gov/Blast.cgi}) \citep{camacho2009}
    provide a means to investigate the evolution of one or a few genes via
    similarity searching, and automated pipelines such as orthoMCL
    \citep{li2003} attempt to rapidly perform orthology prediction for all
    genes in several genomes. AMOEBAE addresses the problem mid-scale analyses
    which are too cumbersome to be done via webservices and yet requiring a
    level of detail and flexibility not offered by automated pipelines. AMOEBAE
    may be useful for analyzing the distribution of orthologues of up to
    perhaps 30 genes/proteins among a sampling of no more than approximately
    100 eukaryotic genomes.  However, you may need to carefully define the
    scope of your analysis depending on what additional steps you may find
    necessary beyond those that may be performed using AMOEBAE (30 queries and
    100 genomes may in fact be unmanageable). AMOEBAE provides many options
    which can be tailored to the specific genes/proteins being analyzed, and
    allow analyses using complex sets of customized criteria to be reproduced
    more practically. 

    %Moreover, it should be clear that AMOEBAE identifies "positive" and
    %"negative" results simply by applying criteria that the user specifies. So,
    %it is entirely the users responsibility to select appropriate criteria and
    %interpret the results critically. 

\subsection{Key features}


The core functionality is to run sequence similarity searches with multiple
    algorithms, multiple queries, and multiple databases simultaneously in
    parallel and to facilitate efficient and highly customizable implementation
    of reciprocal-best-hit search strategies. The output includes detailed
    summaries of results in the form of a spreadsheet and plots.
    
    A particular advantage of AMOEBAE over other tools is the functionality for
    parsing results of TBLASTN (searching in nucleotide sequences with peptide
    sequence queries) search results. This allows rapid identification of
    High-scoring Segment Pair (HSP) clusters at separate gene loci (identified
    according to user-defined criteria), automatic checking of those loci
    against information in genome annotation files, and systematic use of
    Exonerate \citep{slater2005} where possible for obtaining better exon
    predictions.  

\subsection{User support}

For specific issues with the code, please use the issue tracker on the GitHub
    webpage here: \url{https://github.com/laelbarlow/amoebae/issues}. 

If you have general questions regarding AMOEBAE, please email the author at
    lael (at) ualberta.ca.

\subsection{Documentation}

This document provides an overview of AMOEBAE and describes the functionality
of the various commands/scripts. For a tutorial which includes a working
example of a similarity search analysis run using AMOEBAE, see the Jupyter
Notebook: amoebae/notebooks/similarity\_search\_tutorial\_2.ipynb. For code
documentation, please see the html file(s), which can be opened with your web
browser: \url{amoebae/doc/code_documentation/html/index.html}.


\subsection{How to cite AMOEBAE}

Please cite the GitHub webpage \url{https://github.com/laelbarlow/amoebae} (or
alternative permanent repositories if relevant). Also, the first publication to
make use of a version of AMOEBAE was an analysis of Adaptor Protein subunits in
embryophytes by \cite{larson2019}.

Also, you may wish to cite the software packages which are key dependencies of
AMOEBAE, since AMOEBAE would not work without these (see
\autoref*{dependencies_section}).
% Need to insert a mini bibliography here?

\subsection{Acknowledgments}

\subsection{License}
\label{license_section}

Copyright 2018 Lael D. Barlow

Licensed under the Apache License, Version 2.0 (the "License"); you may not use this file except in compliance with the License. You may obtain a copy of the License at

\url{http://www.apache.org/licenses/LICENSE-2.0}

Unless required by applicable law or agreed to in writing, software distributed under the License is distributed on an "AS IS" BASIS, WITHOUT WARRANTIES OR CONDITIONS OF ANY KIND, either express or implied. See the License for the specific language governing permissions and limitations under the License.


\section{How to start using AMOEBAE}

\subsection{System requirements}

Please note that the commands shown likely only work on macOS or Linux operating
systems (you may have trouble running AMOEBAE directly on Windows). 


\subsection{Dependencies}
\label{dependencies_section}

All dependencies are free and open-source, and can be automatically installed
in a virtual environment (see \autoref*{setup_section}).

These are the main depencencies of AMOEBAE:

\begin{itemize}

\item Python3 (the Anaconda distribution works well).

\item Biopython, a Python package for bioinformatics \citep{cock2009}.

\item The Environment for Tree Exploration 3 (ETE3), a Python package for
    working with phylogenetic trees \citep{huerta-cepas2016}.

\item Matplotlib, a Python package for generating plots \citep{hunter2007}.

\item (\href{https://pythonhosted.org/gffutils/}{gffutils}).

\item NCBI BLAST+, a software package for sequence similarity searching \citep{camacho2009}.

\item HMMer3, a software package for profile sequence similarity searching \citep{eddy1998}.

\item MUSCLE, for multiple sequence alignment \citep{edgar2004}.

\item IQ-TREE, for phylogenetic analysis \citep{nguyen2015}.


\end{itemize}


\subsection{Setting up an environment for AMOEBAE using Docker}
\label{setup_section}

%Note that currently ETE3, a dependency of AMOEBAE does not work properly with
%Python 3.7, so Python 3.6 will be installed in the virtual environment.

Follow the steps below to set up AMOEBAE on your personal computer. This setup
process will take approximately 1 hour to complete, however, most of the
process is automated, so only about 20 minutes or less is required for the
steps that require manual input.  Instructions for setting up AMOBEAE on a
remote server will soon be added as well.

\begin{enumerate}

\item Ensure that Git is installed on your computer This program may already be
    installed by default on your operating system. If you have a newer version
        of macOS it may prompt you to install Git. Documentation for Git is
        available here: \url{https://git-scm.com/doc}. You can check which
        version you have by running the command below.

\begin{lstlisting}
>>> git --version
\end{lstlisting}

\item Clone the AMOEBAE repository using Git. If you simply download the code
    from GitHub, instead of cloning the repository, then AMOEBAE cannot record
        specifically what version of the code you use, and will not run
        properly. Make sure to use the appropriate directory path (the path
        shown is just an example).  Please note: Here "\texttt{>{}>{}>}" is
        used to indicate that the following text in the line is to be entered
        in you terminal command prompt.

\begin{lstlisting}
>>> cd /path/to/directory/where/you/keep/scripts
>>> git clone https://github.com/laelbarlow/amoebae.git
\end{lstlisting}

%\item Add the directory to your \$PATH variable for access via the command line
%(as well as the misc\_scripts directory, if you wish to access those files). Do
%this by appending a customized version of the following line to your
%        .bash\_profile file.

%\textbf{Warning:} make sure to use '\texttt{>{}>}' rather than '\texttt{>}'
%otherwise you will overwrite any existing lines in your .bash\_profile file.
%
%\begin{lstlisting}
%>>> cd 
%>>> echo 'export PATH=/path/to/directory/where/you/keep/scripts/amoebae:$PATH' >> .bash_profile
%\end{lstlisting}

%\begin{lstlisting}
%export PATH=/path/to/directory/where/you/keep/scripts/amoebae:$PATH
%\end{lstlisting}


\item Make a copy of the settings.py.example file as settings.py. This will be
    customized later. 

\begin{lstlisting}
>>> cd amoebae
>>> cp settings.py.example settings.py
\end{lstlisting}


\item Download and install the appropriate version of Docker from this website:
    \url{https://www.docker.com/products/docker-desktop}.


\item Add the amoebae directory to the list of directories that can be shared
    with Docker containers using the Docker graphical user interface by
    selecting Preferences > Resources > File sharing.


\item Customize the CPUs, memory, etc. that you wish to make available to
    docker containers using the Docker graphical user interface by
    selecting Preferences > Resources > Advanced.


\item Build a Docker image (virtual environment) using the build\_env.sh
    script. This uses the continuumio/anaconda3 image from DockerHub
    (\sloppy\url{https://hub.docker.com/r/continuumio/anaconda3}), and extends it by
    downloading and installing several software packages that AMOEBAE depends
    on. The details of this process are defined in the Dockerfile file in
    the amoebae repository. This step will take approximately 40 minutes to
    complete.

\begin{lstlisting}
>>> bash build_env.sh
\end{lstlisting}


\item Run the Docker using the run\_env.sh script. This generates a Docker
    container from the Docker image built in the preceding step. 

\begin{lstlisting}
>>> bash run_env.sh
\end{lstlisting}


\item Copy and past the resulting URL (the one at the bottom of the output)
    into the address bar of your web browser (either Firefox, Chrome, or Safari
    will work). This should launch a Jupyter sesssion with an interface where
    you can navigate within the amoebae directory from inside the Docker
    container.  For the purposes of the operating system running in the Docker
    container, the amoebae directory is an external volume, like an external
    hard drive. Documentation on Jupyter is available here:
    \url{https://jupyter-notebook.readthedocs.io/en/stable/}. 



\item Click on the "notebooks" directory to open it. Then open one of the
    tutorial files (.ipynb). These Jupyter notebooks include information on how
    to use them once opened.


\item If you wish to save the output displayed within the notebook(s) after
    running, then it may be useful to save a copy of the notebook or save it in
    HTML or PDF format.


\item To shut down the Jupyter session, click the logout button in the jupyter
    browser tabs, and then go to the terminal window that you used to startup
    the Docker container, and press CTRL-C to kill the Jupyter kernel. This
    will close the Jupyter notebooks, but the analysis output files will
    remain, because they are saved to your amoebae/notebooks folder which is on
    your host machine and accessed from within the container.


\end{enumerate}

\section{Command reference}

%\item Enter the following command in a terminal window, which should display a
%list of commands that you can use with amoebae:
%
%\begin{lstlisting}
%>>> amoebae -h
%\end{lstlisting}
%
%You can access further information about each listed command with the -h option
%as follows, and it is recommended that you do this before using each command:
%
%\begin{lstlisting}
%>>> amoebae <commandnamehere> -h
%\end{lstlisting}
%
%To verify that dependencies such as NCBI BLAST are installed, see what is
%printed to standard output upon running the 'check\_depend' command:
%
%\begin{lstlisting}
%>>> amoebae check_depend
%\end{lstlisting}
%
%To verify that modules and python packages such as gffutils can be imported
%successfully by AMOEBAE scripts, use the 'check\_imports' command: 
%
%\begin{lstlisting}
%>>> amoebae check_imports
%\end{lstlisting}

%Import what is printed when amoebae -h is called...

Documentation for each AMOEBAE command and the various options may be accessed
from the command line via the "-h" options. The following command reference
information is the output of running amoebae (and each command) with the "-h"
option.

% Import tex file output from the generate_amoebae_help_output_text.sh.
\input{amoebae_help_output.tex}


\section{Miscellaneous scripts}

see amoebae/misc\_scripts...

%\section{Recommendations regarding similarity searching analysis design}
% (Incorporate into the tutorials)


\printglossaries % https://en.wikibooks.org/wiki/LaTeX/Glossary

\newpage
% End line numbering.
\end{linenumbers}

% The unsrt style orders references by appearance, but puts given names first.
% The plain style orders references alphabetically, but puts surnames first.
\bibliographystyle{laelstyle5} 
\begin{multicols}{2}
{\footnotesize % The footnotesize command makes the text smaller.
\bibliography{references/AMOEBAE}}
\end{multicols}


\end{document}


