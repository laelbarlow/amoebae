\documentclass[12pt,letterpaper]{article}
\usepackage{amsmath}
\usepackage[pdftex]{graphicx}
% Load hyperref package without putting lines around hyperlinks.
%\usepackage[hidelinks]{hyperref}
\PassOptionsToPackage{hyphens}{url}\usepackage[hidelinks]{hyperref}
\usepackage{color} 
\usepackage{xcolor} 
\usepackage{xspace}
\usepackage{anysize}
\usepackage{setspace}
\usepackage{multicol} % This allows multiple columns
\usepackage[nottoc,numbib]{tocbibind} % This makes refs a section
\usepackage[pagewise]{lineno}
\usepackage{tcolorbox} % For making boxes around text.
\usepackage[xindy,toc]{glossaries} % Must come after hyperref package.
%\usepackage{url}
\usepackage[T1]{fontenc} % Makes > not typeset as inverted question mark.

% Remove the "References" header from the bibliography.
\usepackage{etoolbox}
\patchcmd{\thebibliography}{\section*{\refname}}{}{}{}


% Run external commands to generate a tex file containing the output of the
% 'amoebae -h' command.
\immediate\write18{bash generate_amoebae_help_output_text.sh}


% Import stuff for formatting citations.
\usepackage{natbib}

% Format paragraphs.
%\setlength{\parskip}{\baselineskip}%
%\setlength{\parindent}{0pt}%
\setlength{\parindent}{0em}
\setlength{\parskip}{1em}
%\renewcommand{\baselinestretch}{2.0}

% Define command for wrapping large words.
\newcommand*\wrapletters[1]{\wr@pletters#1\@nil}
\def\wr@pletters#1#2\@nil{#1\allowbreak\if&#2&\else\wr@pletters#2\@nil\fi}

% Use the listings package to automatically wrap text (unlike with just using
% verbatim).
\usepackage{listings}
\lstset{
basicstyle=\small\ttfamily,
columns=flexible,
breaklines=true,
keepspaces=true
} 


%% Format section headers (this is not ideal when there are many short
%%subsections. 
%\usepackage[tiny]{titlesec}
%\titlespacing\subsection{0pt}{12pt plus 4pt minus 2pt}{0pt plus 2pt minus 2pt}
%\titlespacing\subsubsection{0pt}{12pt plus 4pt minus 2pt}{0pt plus 2pt minus 2pt}
%\titlespacing\subsubsection{0pt}{12pt plus 4pt minus 2pt}{0pt plus 2pt minus
%2pt}

\marginsize{2.5 cm}{2.5 cm}{1 cm}{1 cm} % Works out to one inch margins.

%\parindent 1cm
\graphicspath{{figures/}}
%\pagenumbering{arabic}

\pagenumbering{roman}

\makeglossaries

\begin{document}
\begin{titlepage}
	\centering
    {\huge AMOEBAE documentation\par}
	\vspace{2cm}
    {\Large Lael D. Barlow\par}
	\vfill
	{\large Version of \today\par}
\end{titlepage}

%Optional table of contents.
\newpage
\tableofcontents

\newpage
\pagenumbering{arabic}
% Start line numbers on this page
\begin{linenumbers}

\section{Introduction}


\subsection{What is AMOEBAE?}

Analysis of MOlecular Evolution with BAtch Entry (AMOEBAE) is a bioinformatics
    software toolkit composed primarily of scripts written in the Python3
    language.  AMOEBAE scripts use existing Python packages including Biopython
    \citep{cock2009}, the Environment for Tree Exploration (ETE3)
    \citep{huerta-cepas2016}, pandas, and Matplotlib \citep{hunter2007} for
    setting up, running, and summarizing analyses of molecular evolution using
    bioinformatics software packages including MUSCLE \citep{edgar2004}, BLAST+
    \citep{camacho2009}, HMMer3 \citep{eddy1998}, and IQ-Tree
    \citep{nguyen2015}. Applications include identifying and classifying
    predicted peptide sequences according to their evolutionary relationships
    with homologues. All dependencies are freely available, and AMOEBAE code is
    open-source (see \autoref*{license_section}) and available on GitHub
    (\url{https://github.com/laelbarlow/amoebae}). 

\subsection{Why use AMOEBAE?}

    Webservices such as those provided by NCBI
    (\url{https://blast.ncbi.nlm.nih.gov/Blast.cgi}) \citep{camacho2009}
    provide a means to investigate the evolution of one or a few genes via
    similarity searching, and automated pipelines such as orthoMCL
    \citep{li2003} attempt to rapidly perform orthology prediction for all
    genes in several genomes. AMOEBAE addresses mid-scale analyses
    which are too cumbersome to be done via webservices and yet require a
    level of detail and flexibility not offered by automated pipelines. AMOEBAE
    may be useful for analyzing the distribution of orthologues of up to
    perhaps 30 genes/proteins among a sampling of no more than approximately
    100 eukaryotic genomes.  However, you may need to carefully define the
    scope of your analysis depending on what additional steps you may find
    necessary beyond those that may be performed using AMOEBAE (30 queries and
    100 genomes may in fact be unmanageable). AMOEBAE provides many options
    which can be tailored to the specific genes/proteins being analyzed, and
    allow analyses using complex sets of customized criteria to be reproduced
    more practically. 

    %Moreover, it should be clear that AMOEBAE identifies "positive" and
    %"negative" results simply by applying criteria that the user specifies. So,
    %it is entirely the users responsibility to select appropriate criteria and
    %interpret the results critically. 


\subsection{Key features}

The core functionality is to run sequence similarity searches with multiple
    algorithms, multiple queries, and multiple databases simultaneously
    and to allow highly customizable implementation
    of reciprocal-best-hit search strategies. The output includes detailed
    summaries of results in the form of a spreadsheet and plots.
    
    A particular advantage of AMOEBAE over other tools is the functionality for
    parsing results of TBLASTN (searching in nucleotide sequences with peptide
    sequence queries) search results. This allows rapid identification of
    High-scoring Segment Pair (HSP) clusters at separate gene loci (identified
    according to user-defined criteria), automatic checking of those loci
    against information in genome annotation files, and systematic use of
    Exonerate \citep{slater2005} where possible for obtaining better exon
    predictions.  


\subsection{A word of caution}

AMOEBAE is not optimized for ease of use, but is meant to be highly
configurable. The many options available to AMOEBAE users inevitably provide
many opportunities for errors in specifying search criteria, and errors in
interpreting output files.  Some prior experience with similarity searching and
with running software using the command line is essential for using AMOEBAE,
and experience writing scripts in Bash and Python would be highly advantageous. 
Moreover, AMOEBAE is still under active development, so some features may not
yet be thoroughly tested.


\subsection{User support}

For specific issues with the code, please use the issue tracker on the GitHub
    webpage here: \url{https://github.com/laelbarlow/amoebae/issues}. 

If you have general questions regarding AMOEBAE, please email the author at
    lael (at) ualberta.ca.

\subsection{Documentation}

This document provides an overview of AMOEBAE and describes the functionality
of the various commands/scripts. For a tutorial which includes a working
example of a similarity search analysis run using AMOEBAE, see the Jupyter
Notebook: amoebae/notebooks/similarity\_search\_tutorial\_2.ipynb. For code
documentation, please see the html file(s), which can be opened with your web
browser: \url{amoebae/doc/code_documentation/html/index.html}.


\subsection{How to cite AMOEBAE}

Please cite the GitHub webpage \url{https://github.com/laelbarlow/amoebae} (or
alternative permanent repositories if relevant). Also, the first publication to
make use of a version of AMOEBAE was an analysis of Adaptor Protein subunits in
embryophytes by \cite{larson2019}.

Also, you may wish to cite the software packages which are key dependencies of
AMOEBAE, since AMOEBAE would not work without these (see
\autoref*{dependencies_section}).
% Need to insert a mini bibliography here?

\subsection{Acknowledgments}

AMOEBAE was initially developed at the Dacks Laboratory at the University of
Alberta, and was supported by National Sciences and Engineering Council of
Canada (NSERC) Discovery grants RES0021028, RES0043758, and RES0046091 awarded
to Joel B. Dacks, as well as an NSERC Postgraduate Scholarship-Doctoral awarded
to Lael D. Barlow.

We acknowledge the support of the Natural Sciences and Engineering Research Council of Canada (NSERC).

Cette recherche a \'et\'e financ\'ee par le Conseil de recherches en sciences naturelles et en g\'enie du Canada (CRSNG).

Also, help with testing AMOEBAE has been kindly provided by Raegan T. Larson,
Shweta V. Pipalya, Kira More, and Krist\'ina Z\'ahonov\'a.

\subsection{License}
\label{license_section}

Copyright 2018 Lael D. Barlow

Licensed under the Apache License, Version 2.0 (the "License"); you may not use this file except in compliance with the License. You may obtain a copy of the License at

\url{http://www.apache.org/licenses/LICENSE-2.0}

Unless required by applicable law or agreed to in writing, software distributed under the License is distributed on an "AS IS" BASIS, WITHOUT WARRANTIES OR CONDITIONS OF ANY KIND, either express or implied. See the License for the specific language governing permissions and limitations under the License.


\section{How to start using AMOEBAE}

\subsection{System requirements}

Please note that the commands shown likely only work on macOS or Linux operating
systems (you may have trouble running AMOEBAE directly on Windows). 


\subsection{Dependencies}
\label{dependencies_section}

All dependencies are free and open-source, and can be automatically installed
in a virtual environment (see \autoref*{setup_section}).

These are the main depencencies of AMOEBAE:

\begin{itemize}

\item Python3 (the Anaconda distribution works well).

\item Biopython, a Python package for bioinformatics \citep{cock2009}.

\item The Environment for Tree Exploration 3 (ETE3), a Python package for
    working with phylogenetic trees \citep{huerta-cepas2016}.

\item Matplotlib, a Python package for generating plots \citep{hunter2007}.

\item (\href{https://pythonhosted.org/gffutils/}{gffutils}).

\item NCBI BLAST+, a software package for sequence similarity searching \citep{camacho2009}.

\item HMMer3, a software package for profile sequence similarity searching \citep{eddy1998}.

\item MUSCLE, for multiple sequence alignment \citep{edgar2004}.

\item IQ-TREE, for phylogenetic analysis \citep{nguyen2015}.


\end{itemize}


\subsection{Setting up an environment for AMOEBAE using Singularity}
\label{setup_section}

%Note that currently ETE3, a dependency of AMOEBAE does not work properly with
%Python 3.7, so Python 3.6 will be installed in the virtual environment.

Follow the steps below to set up AMOEBAE on your personal computer. This setup
process should take approximately 20 minutes to complete. Additional instructions for
setting up AMOEBAE on a remote server will soon be added as well.

\begin{enumerate}

\item Ensure that Git is installed on your computer If you do not already have
    git installed, then your computer will prompt you with instructions for how
        to install it when you type git into the command line. If you have a
        newer version of macOS it may prompt you to install developer tools,
        which may take up a considerable amount of storage space. Documentation
        for Git is available here: \url{https://git-scm.com/doc}. You can check
        which version you have (or whether it is installed at all) by running
        the command below. Please note: Here "\texttt{>{}>{}>}" is
        used to indicate that the following text in the line is to be entered
        in you terminal command prompt. 

\begin{lstlisting}
>>> git --version
\end{lstlisting}

\item Clone the AMOEBAE repository using Git. If you simply download the code
    from GitHub, instead of cloning the repository, then AMOEBAE cannot record
        specifically what version of the code you use, and will not run
        properly. Make sure to use the appropriate directory path (the path
        shown is just an example). Also, replace the path shown below with
        the path to the directory on your system where you wish to put the main
        AMOEBAE directory.

\begin{lstlisting}
>>> cd /path/to/directory/where/you/keep/files
>>> git clone https://github.com/laelbarlow/amoebae.git
\end{lstlisting}

\item Set up AMOEBAE. This performs several steps including checking for
    whether singularity is installed and attempting to use VirtualBox and
        Vagrant to run Singularity in a pre-built Ubuntu virtual machine with
        Singularity installed. This is because Singularity does not run on
        MacOS (or Windows), and installation of Singularity on Linux is
        complex, as several dependencies are required. This script downloads
        a pre-built singularity container, which was built using the
        singularity.recipe file, and provided on the Singularity Library
        (\url{https://cloud.sylabs.io/library/_container/5e8ca8fff0f8eb90a8a7b60d}).

\begin{lstlisting}
>>> cd amoebae
>>> bash setup.sh
\end{lstlisting}


\item If you are setting up AMOEBAE on a high performance computing cluster,
    then you will not be able to install Singularity yourself, and may need to
        use specific procedures to load Singularity prior to use.


\end{enumerate}

\subsection{Running AMOEBAE using Jupyter notebooks}
\label{jupyter_section}


\begin{enumerate}

\item After setting up AMOEBAE according to the instructions above, the easiest
    way to start running analyses using AMOEBAE is via the tutorials, which are in the
        form of Jupyter notebooks (\url{https://jupyter.org/}).  These Jupyter
        notebooks can be run using the installation of Jupyter in the
        Singularity container, and can be accessed using your browser (on a
        personal computer). To start a Jupyter server, run the bash script as
        indicated below (assuming your current working directory is the main
        amoebae directory that you cloned with Git).

\begin{lstlisting}
>>> bash singularity_jupyter.sh
\end{lstlisting}


\item Copy and past the resulting URL (the one at the bottom of the output)
    into the address bar of your web browser (either Firefox, Chrome, or Safari
    will work). This will open Jupyter to the notebooks subdirectory, which
        contains several tutorial and example notebooks (.ipynb files). These
        files are the files on your regular (host) filesystem, as the amoebae
        directory is synced with the Singularity container. Thus changes to
        files will persist after you shut down the Jupyter server and the
        Singularity container. Documentation on Jupyter is available here:
    \url{https://jupyter-notebook.readthedocs.io/en/stable/}. 


\item Click on one of the tutorial files (.ipynb). These Jupyter notebooks
    include information on how to use them once opened. The first tutorial
        (amoebae\_tutorial\_1.ipynb) provides a simple example of similarity
        searching with BLASTP using a Jupyter notebook. The second tutorial
        (amoebae\_tutorial\_2.ipynb) provides an example using most of the
        similarity searching functionality that AMOEBAE provides. 

\item To shut down the Jupyter server, click the logout button in the jupyter
    browser tab(s), and then go to the terminal window that you used to startup
    the Jupyter server, and press CTRL-C to kill the Jupyter kernel. This
    will close the Jupyter notebooks, but the analysis output files will
    remain, because they are saved to your amoebae/notebooks folder which is on
    your host machine and accessed from within the container.


\item Working with the Jupyter notebooks interactively in this manner on
    high-performance computing clusters is likely possible but inconvenient,
        and procedures will vary. Also, running the tutorial notebooks would
        require access to the internet from compute nodes (as opposed to login
        nodes) which may not be supported. Therefore, it is recommended that
        you run the tutorials on a personal laptop/desktop computer if
        possible. To run your own notebooks on a cluster, you will need to
        write a job submission script that will be specific to the cluster, the
        job scheduler it uses, and your account details. Please refer to
        documentation provided by your system administrators for this. For an
        example script that writes a script for running a notebook as a job to
        a SLURM job scheduler see
        \url{https://github.com/laelbarlow/amoebae/blob/master/notebooks/write_notebook_slurm_script.sh}.
        

\end{enumerate}


\subsection{Running AMOEBAE via the command line}
\label{shell_section}

\begin{enumerate}

\item The easiest way to access AMOEBAE dependencies via the command line is to
    use the bash script provided. If you are running singularity in a virtual
        machine (\textit{e.g.}, on MacOS), then only one shell session may be
        opened at once (and these cannot be opened at the same time as the
        singularity\_jupyter.sh script is running Singularity in a virtual
        machine). Running the script as indicated below will open a shell
        session in the Singularity container, with the amoebae directory being
        the only one accessible. Also, the amoebae executable script is added
        to the \$PATH in the container, so you can run amoebae commands from
        any directory.

\begin{lstlisting}
>>> bash singularity_shell.sh
\end{lstlisting}

\item You may find it useful to explore and test the environment using the
    following commands.

\begin{itemize}

\item Print the paths included in the \$PATH variable in the container.
\begin{lstlisting}
>>> tr ':' '\n' <<< "$PATH"
\end{lstlisting}

\item Check the location of the amoebae executable being run from within
    the container.
\begin{lstlisting}
>>> command -v amoebae
\end{lstlisting}

\item Check that the amoebae executable script can be run (print the help
    message).
\begin{lstlisting}
>>> amoebae -h
\end{lstlisting}

\item Check that all modules can be imported in all python files in the
    AMOEBAE code.
\begin{lstlisting}
>>> amoebae check_imports
\end{lstlisting}

\item Check that key dependencies such as BLASTP can be accessed (they are
    installed in the Singularity container).
\begin{lstlisting}
>>> amoebae check_depend
\end{lstlisting}

\end{itemize}

\item Again, running AMOEBAE commands on high-performance computing clusters
    will require you to write custom job submission scripts. Please refer to
        documentation provided by your system administrator(s) regarding
        details specific to your cluster, including the job scheduler used.
        Also, refer to the Singularity documentation for formulating
        Singularity commands (\url{https://sylabs.io/docs/}).

% *** Write section on how to write scripts to run ameobae code on HPC clusters
% (without using Jupyter)...

\end{enumerate}

\section{Command reference}

Documentation for each AMOEBAE command and the various options may be accessed
from the command line via the "-h" options. The following command reference
information is the output of running amoebae (and each command) with the "-h"
option.

% Import tex file output from the generate_amoebae_help_output_text.sh.
\input{amoebae_help_output.tex}


\section{Miscellaneous scripts}

Several scripts of less general applicablity than the amoebae commands descibed
above are included in the AMOEBAE toolkit. See the amoebae/misc\_scripts
directory
(\url{https://github.com/laelbarlow/amoebae/tree/master/misc_scripts}). Most
scripts have information regarding usage in the files themselves. More detailed
information regarding some of these scripts may be added to this documentation
in the future.

%\section{Recommendations regarding similarity searching analysis design}
% (Incorporate into the tutorials)


\printglossaries % https://en.wikibooks.org/wiki/LaTeX/Glossary

\newpage
% End line numbering.
\end{linenumbers}

% The unsrt style orders references by appearance, but puts given names first.
% The plain style orders references alphabetically, but puts surnames first.
\bibliographystyle{laelstyle5} 
\begin{multicols}{2}
{\footnotesize % The footnotesize command makes the text smaller.
\bibliography{references/AMOEBAE}}
\end{multicols}


\end{document}


